%
% this file is encoded in utf-8
% v1.7
% do not change the content of this file
% unless the thesis layout rule is changed
% 無須修改本檔內容,除非校方修改了
% 封面、書名頁、中文摘要、英文摘要、誌謝、目錄、表目錄、圖目錄、符號說明
% 等頁之格式
% this file is encoded in utf-8
%v1.7

% make the line spacing in effect
\renewcommand{\baselinestretch}{\mybaselinestretch}
\large % it needs a font size changing command to be effective
 
 \newcommand\cTitle{}
\newcommand\eTitle{}
\newcommand\myCname{}
\newcommand\myEname{}
\newcommand\myStudentID{}
\newcommand\advisorCnameA{}
\newcommand\advisorEnameA{}
\newcommand\advisorCnameB{}
\newcommand\advisorEnameB{}
\newcommand\advisorCnameC{}
\newcommand\advisorEnameC{}
\newcommand\univCname{}
\newcommand\univEname{}
\newcommand\deptCname{}
\newcommand\fulldeptEname{}
\newcommand\deptEname{}
\newcommand\collEname{}
\newcommand\degreeCname{}
\newcommand\degreeEname{}
\newcommand\cYear{}
\newcommand\cMonth{}
\newcommand\cDay{}
%\newcounter{eYear}
\newcommand\eYear{}
\newcommand\eMonth{}
\newcommand\ePlace{}

%
% this file is encoded in utf-8
% v1.7
% 填入你的論文題目、姓名等資料
% 如果題目內有必須以數學模式表示的符號,請用 \mbox{} 包住數學模式,如下範例
% 如果中文名字是單名,與姓氏之間建議以全形空白填入,如下範例
% 英文名字中的稱謂,如 Prof. 以及 Dr.,其句點之後請以不斷行空白~代替一般空白,如下範例
% 如果你的指導教授沒有如預設的三位這麼多,則請把相對應的多餘教授的中文、英文名
%    的定義以空的大括號表示
%    如,\renewcommand\advisorCnameB{}
%          \renewcommand\advisorEnameB{}
%          \renewcommand\advisorCnameC{}
%          \renewcommand\advisorEnameC{}

% 論文題目 (中文)
\renewcommand\cTitle{
在第一人稱視角下使用時間聚合技術做動作辨識
}

% 論文題目 (英文)
\renewcommand\eTitle{%My Thesis Title  
Temporal Aggregation for Action Recognition in First-Person Video
}

% 我的姓名 (中文)
\renewcommand\myCname{}

% 我的姓名 (英文)
\renewcommand\myEname{Didik~Purwanto}

%我的學號
\renewcommand\myStudentID{M10402804}
% 指導教授A的姓名 (中文)
\renewcommand\advisorCnameA{}

% 指導教授A的姓名 (英文)
\renewcommand\advisorEnameA{Professor Wen-Hsien Fang}

% 指導教授B的姓名 (中文)
\renewcommand\advisorCnameB{}

% 指導教授B的姓名 (英文)
\renewcommand\advisorEnameB{Professor Yie-Tarng Chen}
% 指導教授C的姓名 (中文)
\renewcommand\advisorCnameC{}

% 指導教授C的姓名 (英文)
\renewcommand\advisorEnameC{}

% 校名 (中文)
\renewcommand\univCname{國立台灣科技大學}

% 校名 (英文)
%\renewcommand\univEname{National Taiwan University of science and technology}

% 系所名 (中文)
\renewcommand\deptCname{電~子~工~程~系}

% 系所全名 (英文)
%\renewcommand\fulldeptEname{Graduate School of Electro-Optical Engineering}

% 系所短名 (英文, 用於書名頁學位名領域)
%\renewcommand\deptEname{Electro-Optical Engineering}

% 學院英文名 (如無,則以空的大括號表示)
%\renewcommand\collEname{College of Electrical and Communication Engineering}

% 學位名 (中文)
\renewcommand\degreeCname{碩士}

% 學位名 (英文)
%\renewcommand\degreeEname{Master of Science}

% 口試年份 (中文、民國)
\renewcommand\cYear{ 106 }

% 口試月份 (中文)
\renewcommand\cMonth{ 7 } 

% 口試月份 (中文)
\renewcommand\cDay{ 23 } 


% 學校所在地 (英文)
\renewcommand\ePlace{Taipei, Taiwan}

%畢業級別;用於書背列印;若無此需要可忽略
\newcommand\GraduationClass{}

%%%%%%%%%%%%%%%%%%%%%%


\newcommand\itsempty{}


%%%%%%%%%%%%%%%%%%%%%%%%%%%%%%%
%       ntust cover 封面
%%%%%%%%%%%%%%%%%%%%%%%%%%%%%%%


\begin{titlepage}
% no page number
% next page will be page 1

% aligned to the center of the page
\begin{center}

\begin{figure}[htbp]
	\begin{minipage}[b]{5cm} 
	\raggedright
	\includegraphics[width=1in]{figures/ntust_logo.jpg}
	\label{fig:ntust_logo}
	\end{minipage}% 
	\begin{minipage}[b]{0.5\textwidth} 
	\centering
	\makebox[3cm][c]{\Huge{\univCname}}\\  %顯示中文校名
	\vspace{0.5cm}
	\makebox[3cm][c]{\Huge{\deptCname}}\\ %顯示中文系所名
	\vspace{0.5cm}
	\end{minipage}%
\\ 
\rule{16cm}{3pt}
\end{figure}
\hfill


\vspace{2.5cm}
\makebox[6cm][s]{\textbf{\Huge{\degreeCname 學位論文 }}}\\ %顯示論文種類 (中文)
\vspace{2.5cm}
%
% Set the line spacing to single for the titles (to compress the lines)
\renewcommand{\baselinestretch}{1}   %行距 1 倍
%\large % it needs a font size changing command to be effective
\Large{\cTitle}\\  % 中文題目
%
\vspace{1cm}
%
\Large{\eTitle}\\ %英文題目
\vspace{1cm}
% \makebox is a text box with specified width;
% option s: stretch
% use \makebox to make sure
% 「研究生:」 與「指導教授:」occupy the same width
\vspace{1cm} 
\Large{\myEname}  % 顯示作者中文名

\vspace{0.5cm} 
\Large{\myStudentID}  %顯示指導教授A中文名

\vspace{1cm}
%

 \hspace{1cm}\makebox[3cm][s]{\Large{指導教授:}}
\Large{\advisorEnameA}  %顯示指導教授A中文名
 \makebox[1cm][s]{}\\
%
% 判斷是否有共同指導的教授 B
\ifx \advisorEnameB  \itsempty
\relax % 沒有 B 教授,所以不佔版面,不印任何空白
\else
% 共同指導的教授 B
\hspace{4.5cm} \makebox[0.95 cm][s]{}
\Large{\advisorEnameB}  %顯示指導教授B中文名
\hfill \makebox[2cm][s]{}\\
\fi
%
% 判斷是否有共同指導的教授 C
\ifx \advisorCnameC  \itsempty
\relax % 沒有 C 教授,所以不佔版面,不印任何空白
\else
% 共同指導的教授 C
\hspace{4.5cm} \makebox[3cm][s]{}
\Large{\advisorCnameC}  %顯示指導教授C中文名
\hfill \makebox[1cm][s]{}\\
\fi
%
\vfill
\makebox[10cm][s]{\Large{中華民國\cYear 年\cMonth 月\cDay 日}}%
%
\\




\end{center}
% Resume the line spacing to the desired setting
\renewcommand{\baselinestretch}{\mybaselinestretch}   %恢復原設定
% it needs a font size changing command to be effective
% restore the font size to normal
\normalsize
\end{titlepage}

%%%%%%%%%%%%%%%

%% 從摘要到本文之前的部份以小寫羅馬數字印頁碼
% 但是從「書名頁」(但不印頁碼) 就開始計算
\setcounter{page}{1}
\pagenumbering{roman}
%%%%%%%%%%%%%%%%%%%%%%%%%%%%%%%
%       指導教授推薦書 
%%%%%%%%%%%%%%%%%%%%%%%%%%%%%%%
%

% insert the printed standard form when the thesis is ready to bind
% 在口試完成後,再將已簽名的推薦書放入以便裝訂
% create an entry in table of contents for 推薦書
% 目前送出空白頁

%\newpage{\thispagestyle{empty}\addcontentsline{toc}{chapter}{\nameInnerCover}\mbox{}\clearpage}
%\newpage

%教授推薦書\newgeometry{tmargin=0.1cm,bmargin=0.1cm,lmargin=1cm,rmargin=0.1cm}
%\begin{flushleft}
%\includegraphics[width=\textwidth]{frontpages/thesis_form1.jpeg}
%\end{flushleft}



% 判斷是否要浮水印?
\ifx\mywatermark\undefined 
  \thispagestyle{empty}  % 無頁碼、無 header (無浮水印)
\else
  \thispagestyle{EmptyWaterMarkPage} % 無頁碼、有浮水印
\fi



%%%%%%%%%%%%%%%%%%%%%%%%%%%%%%%%
%%       英文摘要 
%%%%%%%%%%%%%%%%%%%%%%%%%%%%%%%%
%
\newpage
\thispagestyle{plain}  % 無 header,但在浮水印模式下會有浮水印

% create an entry in table of contents for 英文摘要
\addcontentsline{toc}{chapter}{\nameEabstract}

% aligned to the center of the page
\begin{center}
% font size:
% \large (14pt) < \Large (18pt) < \LARGE (20pt) < \huge (24pt)< \Huge (24 pt)
% Set the line spacing to single for the names (to compress the lines)
\renewcommand{\baselinestretch}{1}   %行距 1 倍
%\large % it needs a font size changing command to be effective
%\large{\eTitle}\\  %英文題目
%\vspace{0.83cm}
% \makebox is a text box with specified width;
% option s: stretch
% use \makebox to make sure
% each text field occupies the same width
%\makebox[2cm][s]{\large{Student: }}
%\makebox[5cm][l]{\large{\myEname}} %學生英文姓名
%\hfill
%
%\makebox[2cm][s]{\large{Advisor: }}
%\makebox[5cm][l]{\large{\advisorEnameA}} \\ %教授A英文姓名
%
% 判斷是否有共同指導的教授 B
\ifx \advisorCnameB  \itsempty
\relax % 沒有 B 教授,所以不佔版面,不印任何空白
\else
%共同指導的教授B
\makebox[2cm][s]{}
\makebox[5cm][l]{} %%%%%
\hfill
\makebox[2cm][s]{}
\makebox[5cm][l]{\large{\advisorEnameB}}\\ %教授B英文姓名
\fi
%
% 判斷是否有共同指導的教授 C
\ifx \advisorCnameC  \itsempty
\relax % 沒有 C 教授,所以不佔版面,不印任何空白
\else
%共同指導的教授C
\makebox[2cm][s]{}
\makebox[5cm][l]{} %%%%%
\hfill
\makebox[2cm][s]{}
\makebox[5cm][l]{\large{\advisorEnameC}}\\ %教授C英文姓名
\fi
%
%\vspace{0.42cm}
%\large{Submitted to }\large{\fulldeptEname}\\  %英文系所全名
%
%\ifx \collEname  \itsempty
%\relax % 如果沒有學院名 (英文),則不佔版面,不印任何空白
%\else
% 有學院名 (英文)
%\large{\collEname}\\% 學院名 (英文)
%\fi
%
%\large{\univEname}\\  %英文校名
%\vspace{0.83cm}
%\vfill
%
\large{\textbf{Abstract}}\\
\vspace{0.5cm}
\end{center}
% Resume the line spacing the desired setting
\renewcommand{\baselinestretch}{\mybaselinestretch}   %恢復原設定
%\large %it needs a font size changing command to be effective
% restore the font size to normal
\normalsize
%%%%%%%%%%%%%


\blue{
First person point of view video is a video produced by 
egocentic camera that capture the scene with first person perspective. 
The characteristic of the first-person point of view video 
in general features are rich in noise and contain plenty of uncontrolled variations scene.
Moreover, these videos need to deal with some adverse effects such as shaky frame, background clutter, occlusion and inter-class variation. Coupled with the absence of the actor's pose, 
it makes the first-person video is more challenging than the third person video.
}

This \blue{thesis} presents a new approach for action recognition in
first-person videos which aggregates both of the short- and
long-term trends based on the coefficients of the Hilbert-Huang
transform (HHT), a renowned time-frequency analysis tool. In
contrast to previous works like Pooled Time Series (PoT), the new
approach can extract the salient features of activities based on the
non-stationary HHT analysis, which consists of empirical mode
decomposition and Hilbert spectral analysis. The proposed approach
can be incorporated with the convolutional neural network (CNN)
features such as trajectory pooled CNN features to
achieve superior detection accuracy. Simulations show that
the proposed method outperforms the main state-of-the-art works on
\blue{three} widespread public first-person datasets.

\textbf{Keywords}: temporal pyramid pooling, temporal
aggregation, first-person video, action recognition, Hilbert-Huang transform


%%%%%%%%%%%%%%%%%%%%%%%%%%%%%%%%
%%       Publications
%%%%%%%%%%%%%%%%%%%%%%%%%%%%%%%%


\newpage
\chapter*{\protect\makebox[5cm][s]{Related Publications}} 
\addcontentsline{toc}{chapter}{Related Publications}


%\textbf{Journal}
%\begin{itemize}
%    \item  \textbf{Didik Purwanto}, Yie-Tarng Chen, Wen-Hsien Fang, Erick Hendra Putra Alwando, and Rizard Renanda. ``Deep Learned Features for Action Recognition in First-Person Video". \textit{IEEE Transactions on Circuit and System for Video Technology}, 2017 [submitted]
%    \item  Erick Hendra Putra Alwando, Yie-Tarng Chen, Wen-Hsien Fang, \textbf{Didik Purwanto}, and Rizard Renanda. ``Multi Object Tracking for Action Localization". \textit{IEEE Transactions on Circuit and System for Video Technology}, 2017 [submitted]
%\end{itemize}
%\textbf{Conference}
\begin{itemize}
    \item \textbf{Didik Purwanto}, Yie-Tarng Chen, and Wen-Hsien Fang. ``Temporal Agregation for First-Person Video using Hilbert-Huang Transform", \textit{The IEEE Conference on Multimedia and Expo (ICME)}, Hongkong, 2017.
    \item Yie-Tarng Chen, Ting-Zhi Wang, Wen-Hsien Fang, and \textbf{Didik Purwanto}. ``Learning Visual Object and Word Association", \textit{The IEEE Conference on Signal and Image Processing and Aplications (ICSIPA)}, Kuching, 2017.
\end{itemize}
%%%%%%%%%%%%%%%%%%%%%%%%%%%%%%%%

%Acknowledgment
\newpage
\chapter*{\protect\makebox[5cm][s]{\nameAckn}} %\makebox{} is fragile; need protect%
\addcontentsline{toc}{chapter}{\nameAckn}

First of all, I am grateful to Allah for establishing me to finish this master thesis. In addition, I would like to thank to my advisers Professor Wen-Hsien, Fang and Professor Yie-Tarng, Chen for all the guidance. Their guidance has been essential  to me  as a budding researcher in computer vision and machine learning. Thank you for pushing me beyond my limit. It was both an honour and a great privilege to work with them.

Studying at NTUST was great for me. I was fortunate to have met brilliant students and wonderful friends, including EE 401-2 lab member, Bapel UT Taiwan, PCI Nahdlatul Ulama Taiwan, and Ngaji Bareng member. Also, my profound gratitude to 1-118 roommates — Adrian Chriswanto, Erick, Alrezza, Hilmil, Nizar, Billy, Farid, and also my partner — Rizard Purnomo, Kai, Her Long, Kevin and everyone who has not been mentioned for providing me with unfailing support and continuous encouragement throughout my years of study and through the process of researching and writing this thesis. 

This journey would not have been possible without Google and StackOverflow, which always provide some suggestions when I got stuck on my work. Finally, I must express my deep gratitude to my parents, brother, and sister who always support me from Indonesia.  This accomplishment would not have been possible without them. Thank you.



% Table of contents
\newpage
\renewcommand{\contentsname}{\protect\makebox[5cm][s]{\nameToc}}
%\makebox{} is fragile; need protect
\addcontentsline{toc}{chapter}{\nameToc}
\tableofcontents

% List of Figures
\newpage
\renewcommand{\listfigurename}{\protect\makebox[5cm][s]{List of Figures}}
%\makebox{} is fragile; need protect
\addcontentsline{toc}{chapter}{List of Figures}
\listoffigures

% List of Tables
\newpage
\renewcommand{\listtablename}{\protect\makebox[5cm][s]{\nameLot}}
%\makebox{} is fragile; need protect
\addcontentsline{toc}{chapter}{\nameLot}
\listoftables



\newpage
\pagenumbering{arabic}

